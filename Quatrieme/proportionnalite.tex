\documentclass{article}

\usepackage[utf8]{inputenc}  % specify the input encoding
% \usepackage[french]{babel} % french typography


\usepackage{amsmath}
\usepackage{amssymb}

\usepackage{enumitem} % for changing questions-numbering
 

\title{Proportionnalité à l'aide d'un tableur}
\author{}
\date{}

\begin{document}

\maketitle

{\bf Niveau :} $4^{\textrm{ème}}$  \\

{\bf Objectifs :}  \\

Utilisation d'un tableur pour : 
\begin{list}{$\bullet$}{}
\item vérifier qu'un tableau est de proportionnalité ;
\item générer des valeurs d'un tableau de proportionnalité.
\end{list}

% Les différentes méthodes pour la proportionnalité vues en 4ème :
% \begin{list}{$\bullet$}{}
% \item coefficient de proportionnalité
% \item règles sur les colonnes
% \item relations entre les colonnes
% \item passage à l'unité ou règle de trois
% \item égalité des produits en croix
% \end{list}


% \section{Rappels sur la proportionnalité}

\section{Coefficient de proportionnalité}

% Définition d'un tableau de variation et du coefficient de proportionnalité :

%   Un tableau est dit de proportionnalité lorsqu'on peut passer de la première ligne à la deuxième ligne en multipliant ou en divisant tous les nombres  par le m\^eme nombre.

% Ce nombre est appelé coefficient de proportionnalité.


Le tableau suivant indique le prix en euros de certaines quantités de pêches en kilogrammes :

\begin{equation*}
  \begin{array}{|c|c|c|c|c|}
    \hline
    & A & B & C & D \\
    \hline
    1 & \textrm{Masse de pêches} & 3 & 4 & 5 \\
    & \textrm{(en kg)} & & & \\
    \hline
    2 & \textrm{Prix} & 8,40 & 11,20 & 14 \\
    & \textrm{(en euros)} & & & \\
    \hline
  \end{array}
\end{equation*}

\begin{enumerate}[leftmargin=0cm,itemindent=.5cm,labelwidth=\itemindent,labelsep=0cm,align=left,label=\arabic*)]

\item Déterminer le coefficient de proportionnalité de ce tableau.

\item On souhaite calculer avec un tableur le coefficient de proportionnalité dans la cellule $B3$. 

\begin{equation*}
  \begin{array}{|c|c|c|c|c|}
    \hline
    & A & B & C & D \\
    \hline
    1 & \textrm{Masse de pêches} & 3 & 4 & 5 \\
    & \textrm{(en kg)} & & & \\
    \hline
    2 & \textrm{Prix} & 8,40 & 11,20 & 14 \\
    & \textrm{(en euros)} & & & \\
    \hline
    3 & \textrm{Coefficient de proportionnalité} & & & \\
    \hline
  \end{array}
\end{equation*}
  
  Indiquer la formule à entrer dans la cellule $B3$ afin d'obtenir le coefficient de proportionnalité du tableau.

\item On souhaite obtenir, avec le tableur, les prix de masse de pêches allant de $6$ à $10$ kg.

\begin{equation*}
  \begin{array}{|c|c|c|c|c|c|c|c|c|c|}
    \hline
    & A & B & C & D & E & F & G & H & I\\
    \hline
    1 & \textrm{Masse de pêches} & 3 & 4 & 5 & 6 & 7 & 8 & 9 & 10 \\
    & \textrm{(en kg)} & & & & & & & &\\
    \hline
    2 & \textrm{Prix} & 8,40 & 11,20 & 14 & & & & &\\
    & \textrm{(en euros)} & & & & & & & &\\
    \hline
    3 & \textrm{Coefficient de proportionnalité} & & & & & & & &\\
    \hline
  \end{array}
\end{equation*}


Quelle formule, peut-on entrer dans la cellule $E2$ afin obtenir les prix correspondants par recopie sur la droite de la plage $E2\negthinspace:I2$ ? 
  

% il faut ajouter les 2 dollars pour appeler la B3 du coefficient de proportionnalité
% =E$1*$B$3

\item Générer le tableau à l'aide d'un tableur.

\end{enumerate}


\section{Produits en croix}

On donne des tableaux 2 * 2

Dire si les tableaux suivants sont des tableaux de proportionnalité

utiliser la structure conditionnelle si

produit en croix

\begin{equation*}
  \begin{array}{|c|c|}
    \hline
    a & b \\
    \hline
    c & d \\
    \hline
  \end{array}
\end{equation*}

\end{document}