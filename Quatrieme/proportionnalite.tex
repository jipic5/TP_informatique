\documentclass{article}

\usepackage[utf8]{inputenc}  % specify the input encoding
\usepackage[french]{babel} % french typography


\usepackage{amsmath}
\usepackage{amssymb}

\usepackage{enumitem} % for changing questions-numbering

\setlength{\hoffset}{-18pt}	
\setlength{\oddsidemargin}{0pt}	% Marge gauche sur pages impaires
\setlength{\evensidemargin}{8pt}	% Marge gauche sur pages paires
\setlength{\marginparwidth}{54pt}	% Largeur de note dans la marge
\setlength{\textwidth}{500pt}	% Largeur de la zone de texte 
\setlength{\voffset}{-18pt}	% Bon pour DOS
\setlength{\marginparsep}{7pt}	% Séparation de la marge
\setlength{\topmargin}{0pt}	% Pas de marge en haut
\setlength{\headheight}{13pt}	% Haut de page
\setlength{\headsep}{10pt}	% Entre le haut de page et le texte
\setlength{\footskip}{27pt}	% Bas de page + séparation
% \setlength{\textheight}{708pt}	% Hauteur de la zone de texte (25cm)\s
\setlength{\textheight}{680pt}	% Hauteur de la zone de texte (25cm)\s



\title{Proportionnalité à l'aide d'un tableur}
\author{}
\date{}

\begin{document}

\maketitle

\noindent {\bf Niveau :} $4^{\textrm{ème}}$  \\

\noindent {\bf Objectifs :}  \\

\noindent Utilisation d'un tableur pour : 
\begin{list}{$\bullet$}{}
\item vérifier qu'un tableau est de proportionnalité ;
\item générer des valeurs d'un tableau de proportionnalité.
\end{list}

% Les différentes méthodes pour la proportionnalité vues en 4ème :
% \begin{list}{$\bullet$}{}
% \item coefficient de proportionnalité
% \item règles sur les colonnes
% \item relations entre les colonnes
% \item passage à l'unité ou règle de trois
% \item égalité des produits en croix
% \end{list}


% \section{Rappels sur la proportionnalité}

\section{Coefficient de proportionnalité}

% Définition d'un tableau de variation et du coefficient de proportionnalité :

%   Un tableau est dit de proportionnalité lorsqu'on peut passer de la première ligne à la deuxième ligne en multipliant ou en divisant tous les nombres  par le m\^eme nombre.

% Ce nombre est appelé coefficient de proportionnalité.


Le tableau suivant indique le prix en euros de certaines quantités de pêches en kilogrammes :

\begin{equation*}
  \begin{array}{|c|c|c|c|c|}
    \hline
    & A & B & C & D \\
    \hline
    1 & \textrm{Masse de pêches} & 3 & 4 & 5 \\
    & \textrm{(en kg)} & & & \\
    \hline
    2 & \textrm{Prix} & 8,40 & 11,20 & 14 \\
    & \textrm{(en euros)} & & & \\
    \hline
  \end{array}
\end{equation*}

\begin{enumerate}[leftmargin=0cm,itemindent=.5cm,labelwidth=\itemindent,labelsep=0cm,align=left,label=\arabic*)]

\item Déterminer le coefficient de proportionnalité de ce tableau.

\item On souhaite calculer avec un tableur le coefficient de proportionnalité dans la cellule $B3$. 

\begin{equation*}
  \begin{array}{|c|c|c|c|c|}
    \hline
    & A & B & C & D \\
    \hline
    1 & \textrm{Masse de pêches} & 3 & 4 & 5 \\
    & \textrm{(en kg)} & & & \\
    \hline
    2 & \textrm{Prix} & 8,40 & 11,20 & 14 \\
    & \textrm{(en euros)} & & & \\
    \hline
    3 & \textrm{Coefficient de proportionnalité} & & & \\
    \hline
  \end{array}
\end{equation*}
  
  Indiquer la formule à entrer dans la cellule $B3$ afin d'obtenir le coefficient de proportionnalité du tableau.

\item On souhaite obtenir, avec le tableur, les prix de masse de pêches allant de $6$ à $10$ kg.

\begin{equation*}
  \begin{array}{|c|c|c|c|c|c|c|c|c|c|}
    \hline
    & A & B & C & D & E & F & G & H & I\\
    \hline
    1 & \textrm{Masse de pêches} & 3 & 4 & 5 & 6 & 7 & 8 & 9 & 10 \\
    & \textrm{(en kg)} & & & & & & & &\\
    \hline
    2 & \textrm{Prix} & 8,40 & 11,20 & 14 & & & & &\\
    & \textrm{(en euros)} & & & & & & & &\\
    \hline
    3 & \textrm{Coefficient de proportionnalité} & & & & & & & &\\
    \hline
  \end{array}
\end{equation*}


Quelle formule, peut-on entrer dans la cellule $E2$ afin obtenir les prix correspondants par recopie sur la droite de la plage $E2\negthinspace:I2$ ? 
  

% il faut ajouter les 2 dollars pour appeler la B3 du coefficient de proportionnalité
% =E$1*$B$3

\item Générer le tableau à l'aide d'un tableur.

\end{enumerate}


\section{Produits en croix}

On considère le tableau suivant indiquant le prix d'un produit dont la quantité est donnée en kg.

% Forme générale : tableau 2*2
% \begin{equation*}
%   \begin{array}{|c|c|}
%     \hline
%     a & b \\
%     \hline
%     c & d \\
%     \hline
%   \end{array}
% \end{equation*}


% tableau de proportionnalité
\begin{equation*}
  \begin{array}{|c|c|c|c|}
    \hline
    & A & B & C \\
    \hline
    1 & \textrm{quantité (en kg)} & 2 & 14 \\
    \hline
    2 & \textrm{prix (en euros)} & 1,5 & 10,5 \\
    \hline
  \end{array}
\end{equation*}

\begin{enumerate}[leftmargin=0cm,itemindent=.5cm,labelwidth=\itemindent,labelsep=0cm,align=left,label=\arabic*)]

\item Montrer en utilisant l'égalité des produits en croix que le prix est proportionnel à la quantité.
  
  % Utilisation de la fonction SI
\item Utiliser la struture conditionnelle SI pour afficher la chaîne de caractère ``proportionnel'' dans la cellule $D4$ lorsque les produits en croix sont égaux.

% Commande sous tableur : 
% =if(B1*C2=B2*C1;"Prop";"non Prop")

\item Utiliser de même la struture conditionnelle SI pour afficher dans la cellule $D4$
  la chaîne de caractères ``proportionnel''  lorsque le tableau est un tableau de proportionnalité
  et ``non proportionnel'' lorsque le tableau n'est pas un tableau de proportionnalité pour les $3$ tableaux suivants.


  \begin{enumerate}[leftmargin=0cm,itemindent=.5cm,labelwidth=\itemindent,labelsep=0cm,align=left,label=\alph*)]
    
  \item 
    % tableau de non-proportionnalité
    \begin{equation*}
      \begin{array}{|c|c|c|c|}
        \hline
        & A & B & C \\
        \hline
        1 & \textrm{quantité (en kg)} & 3 & 3,6 \\
        \hline
        2 & \textrm{prix (en euros)} & 2,3 & 2,53 \\
        \hline
      \end{array}
    \end{equation*}
    
  \item 
    % tableau de non-proportionnalité
    \begin{equation*}
      \begin{array}{|c|c|c|c|}
        \hline
        & A & B & C \\
        \hline
        1 & \textrm{quantité (en kg)} & 7 & 16,1 \\
        \hline
        2 & \textrm{prix (en euros)} & 3,4 & 7,48 \\
        \hline
      \end{array}
    \end{equation*}
    
  \item 
    % tableau de proportionnalité
    \begin{equation*}
      \begin{array}{|c|c|c|c|}
        \hline
        & A & B & C \\
        \hline
        1 & \textrm{quantité (en kg)} & 9 & 23,4 \\
        \hline
        2 & \textrm{prix (en euros)} & 7,6 & 19,76 \\
        \hline
      \end{array}
    \end{equation*}
    
  \end{enumerate}
  
\end{enumerate}


\end{document}